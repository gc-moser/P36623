\documentclass[
11pt,
usepdftitle=false,
aspectratio=169,
xcolor={table,usenames,dvipsnames},
%handout,
]{beamer}

\usetheme[nototalframenumber,nosectiontitlepage,foot,logo,url]{uibk}
\usepackage[utf8]{inputenc}
\usepackage{amsmath,amssymb,alltt}
\usepackage{xcolor}
\usepackage[english]{babel}
\usepackage{stmaryrd}
\usepackage{wasysym}
\usepackage{pifont}
\RequirePackage{etex}
\usepackage{tikz}
\usetikzlibrary{backgrounds}
\usetikzlibrary{fit}
\usetikzlibrary{matrix}
\usetikzlibrary{shapes.symbols}
\usetikzlibrary{shapes.callouts}
\usetikzlibrary{shapes.geometric}
\usetikzlibrary{arrows}
\usetikzlibrary{trees}
\usetikzlibrary{shadows}
\usetikzlibrary{automata}
\usetikzlibrary{decorations.pathmorphing}
\usetikzlibrary{decorations.pathreplacing}
\usetikzlibrary{calc}
\usetikzlibrary{fadings}
\usetikzlibrary{positioning}
\usepackage{mathpartir}
\usepackage{graphicx}
\usepackage{proof}
% \usepackage{bussproofs}
% \usepackage{boxproof}
\usepackage{xargs}
\usepackage{paralist}
\usepackage{multirow}
\usepackage{listings}
\usepackage{turnstile}
\usepackage{xfrac}
\usepackage[most]{tcolorbox}
\usepackage{arydshln}
\usepackage{slides}
\usepackage{code}

\setbeamertemplate{sidebar right}{}
\setbeamertemplate{sections/subsections in toc}[ball unnumbered]
\setbeamersize{text margin left=5mm,text margin right=5mm}
\setbeamertemplate{itemize items}[circle]
\setbeamertemplate{enumerate items}[square]
\setbeamertemplate{qed symbol}{\vrule width 1.5ex height 1.5ex depth 0pt}

\newcommand{\DARKGREEN}[1]{{\color{green!30!black}#1}}
\newcommand{\BLACK}[1]{{\color{black}#1}}
\newcommand{\RED}[1]{{\alert{#1}}}
\newcommand{\DARKRED}[1]{{\color{red!50!black}#1}}
\newcommand{\DARKBLUE}[1]{{\color{blue!50!black}#1}}
\newcommand{\BLUE}[1]{{\color{blue}#1}}
\newcommand{\MAGENTA}[1]{{\color{magenta}#1}}
\newcommand{\YELLOW}[1]{{\color{yellow}#1}}
\newcommand{\WHITE}[1]{{\color{white}#1}}
\definecolor{darkgray}{rgb}{0.31,0.31,0.33}
\newcommand{\GRAY}[1]{{\color{darkgray}#1}}
\definecolor[named]{darkviolet}{HTML}{1b0160}
\definecolor[named]{keyword}{HTML}{e0004a} %TODO: make c/b friendlier
%\definecolor[named]{keyword}{rgb}{0.99,0.78,0.07}
\definecolor[named]{constructor}{HTML}{009304}
\definecolor[named]{warmYellow}{rgb}{0.99,0.78,0.07}
\definecolor[named]{extension}{HTML}{1e88e5}% coin, tick ("probabilistic extension"), nondeterminism
\definecolor[named]{basic}{HTML}{1b0160}

\newcommand{\red}[1]{{\color{orange} #1}} % colorblind friendly
\newcommand{\blue}[1]{{\color{blue!80} #1}}
\newcommand{\green}[1]{{\color{ForestGreen} #1}}

\newcommand*{\yellowemph}[1]{%
\tikz[baseline=(X.base)] \node[rectangle, fill=yellow, fill opacity=0.3, text opacity=1, inner sep=1mm, rounded corners] (X) {#1};%
}

\newcommand*{\boxemph}[1]{%
\tikz[baseline=(X.base)] \node[rectangle, draw, text opacity=1, inner sep=1mm, rounded corners] (X) {#1};%
}

% color scheme for TikZ drawings
%\input{colorscheme}

\newenvironment{mybox}[1]{\begin{block}{#1}}{\end{block}}
\newenvironment{tctsays}{\begin{mybox}{\large \tct\ says}}{\end{mybox}}
\newenvironment{question}[1]{\begin{block}{\MAGENTA{Question~{#1}}}}{\end{block}}
\newenvironment{answer}{\begin{block}{\MAGENTA{Answer}}}{\end{block}}

%GM changes to beamerthemeuibk.sty
\setbeamercolor*{alerted text}{fg=red}

% Listings Configuration
\lstset{%
  language=splay,%
  numbers=left,%
  numberstyle=\tiny\color{black},%
  numbersep=10pt,%
  backgroundcolor=\color{white},%
  frame=single,%
  rulecolor=\color{green!30!black},%
  framerule=0.75pt,%
}

% tcolorbox
\tcbset{colback=white,
  colframe=darkviolet,
  highlight math style= {enhanced, %<-- needed for the ’remember’ options
    colframe=red,colback=red!10!white,boxsep=0pt},
  arc=0mm,
  boxrule=.2mm
}

\title{Automated Sublinear Amortised Resource Analysis of\\[1ex]
  Data Structures (AUTOSARD)}

\author[GM]{Georg Moser \and Florian Zuleger}

\date{December~2, 2024}
\headerimage{4}
\footertext{Kick-Off Workshop, }

\begin{document}
%Z% start every section with a simple frame containing \insertsection
\AtBeginSection[]{%
  \begin{frame}[plain]
    \begin{center}
      %\usebeamercolor[fg]{structure}
      % \huge\bfseries\insertsection
      \title\insertsection
      \URL{}
      \maketitle
      % workaround for repeated use of title slides
\tikz[remember picture,overlay,anchor=north west,inner sep=0pt]{%
\node[xshift=0mm,yshift=0mm] at (current page.north west) {\includegraphics[width=42.1mm]{\mylogoimage}};}
    \end{center}
  \end{frame}
}

%Z% also manually generate title page on a plain frame (w/o footer)
\begin{frame}[plain]
  \titlepage
\end{frame}

\author{}
\subtitle{}

\section{Goals of the Project}
\begin{frame}

  \begin{mybox}{Objectives}
    \begin{enumerate}
    \item[\textbf{A}]<1-> \textbf{Sublinear Amortised Cost Analysis}
      \begin{itemize}
      \item amortised data structures typically feature \alert{sublinear} amortised costs
      \item we want to derive such bounds in a principled and fully automated fashion
      \item we want to build a library of potential functions that can be used for the automated analysis of the most common data structures
      \end{itemize}
      % Amortised data structures--often taken up for their efficiency in practice--typically feature \emph{sublinear} amortised costs.
      % We want to extend our pilot project to derive such bounds in a principled and fully automated fashion.
      % Our pilot project employed only the ``sum of logarithms'' potential function.
      % We want to build a library of potential functions that can be used for the automated analysis of the most common data structures.
      % These potential functions will need to be accompanied by suitable techniques for reasoning about their non-linear behaviour (as was needed for reasoning about the logarithms).

      \smallskip
      
    \item[\textbf{B}]<2-> \textbf{Amortised Cost Analysis for Lazy Evaluation}
      \begin{itemize}
      \item lazy evaluation offers distinct advantages for the design of \alert{persistent} data structures with good amortised computational complexity
      \item we will adapt the techniques from our pilot project to lazy evaluation
      \item this will require to adapt the notion of amortised analysis to \alert{debits} instead of \alert{credits}
      \end{itemize}
      % Resource analysis of non-strict programming languages has been notoriously difficult.
      % However, lazy evaluation offers distinct advantages for the design of \emph{persistent} data structures with good amortised computational complexity as argued by Okasaki~\cite{Okasaki:1999}.
      % In~AUTOSARD, we will adapt the techniques from our pilot project to lazy evaluation.
      % % 
      % % The advantages of amortised data structures immediately breaks if one applies them \emph{persistently}, which, however can be overcome if used in the context of \emph{lazy evaluation}.
      % % 
      % This will require to adapt the notion of amortised analysis to \emph{debits} instead of \emph{credits} as proposed by Okasaki~\cite{Okasaki:1999}.
      
      \smallskip
    \item[\textbf{C}]<3-> \textbf{Expected Cost Analysis for Probabilistic Data Structures}
      \begin{itemize}
      \item the performance gains in the use of probabilistic data structures are well known
      \item we will extend our analyses to further probabilistic data structures
      \end{itemize}
 %      Probabilistic algorithms have been studied since the advent of theoretical computer science.
% Not surprisingly the performance
% gains in the use of probabilistic data structures have been noted early on.
% Motivated by our recent success in the analysis of the expected amortised
% costs of \emph{meldable heaps}, \emph{randomised splay trees} and \emph{randomised splay heaps} in our pilot project,
% we will extend our analysis to further probabilistic data structures.

    \end{enumerate}
  \end{mybox}
\end{frame}

\begin{frame}
  \frametitle{Sublinear Amortised Cost Analysis}
  \vspace{-1ex}
  \begin{itemize}
  \item we want to build a library of potential functions that can be used for the automated analysis of the most common data structures
  \end{itemize}

  \medskip
  \onslide<2->
  
  \ueb{Milestones}
  \begin{itemize}
  \item we consider to confirm the analysis of \BLUE{pairing heaps} due to Iacono, which requires the combination of several potential functions
  \item we will consider \BLUE{skew heaps} and \BLUE{Fibonacci heaps}, \BLUE{union-find data structures}
  \item parts of standard OCaml libraries, like the \texttt{set.ml} library
  \item as well as challenging examples considered in the literature
  \end{itemize}

  \medskip
  \onslide<3->
  
  \ueb{Work in progess}
  \begin{itemize}
  \item we've studied formal amortised cost analyses of \alert{rank-balanced trees}, \alert{binomial heaps}
    in Liquid Haskell; Jamie will later report (I hope) on a formalisation of \alert{Fibonacci heaps}
  \item \atlas\ has been re-implemented, extending the library of potential functions partly (see Armin's talk)
  \end{itemize}
\end{frame}

\begin{frame}
  \frametitle{Amortised Cost Analysis for Lazy Evaluation}
   \vspace{-1ex}
  \begin{itemize}
  \item we will adapt the techniques from our pilot project to lazy evaluation
  \end{itemize}

  \medskip
  \onslide<2->
  
  \ueb{Milestones}
  \begin{itemize}
  \item we consider the analysis of \BLUE{lazy skew heaps}
  \item we will further consider (parts of) standard Haskell libraries like
    eg.\ Haskell's \texttt{containers} library
  \item as well as challenging examples considered in the literatur
  \end{itemize}

  \medskip
  \onslide<3->
  
  \ueb{Work in progess}
  \begin{itemize}
  \item none \smiley
  \end{itemize}

\end{frame}

\begin{frame}
  \frametitle{Expected Cost Analysis for Probabilistic Data Structures}
  \vspace{-1ex}
  \begin{itemize}
  \item we will extend our analyses to further probabilistic data structures
  \end{itemize}

  \medskip
  \onslide<2->
  
  \ueb{Milestones}
  \begin{itemize}
  \item we consider the analysis of \BLUE{random search trees}, which will require the support of non-constant probabilities
  \item Further interesting case studies include \BLUE{skip lists}, \BLUE{randomised treaps}, as well as
    \BLUE{Coupon Collector problem} or \BLUE{randomised quicksort}
  \end{itemize}

  \medskip
  \onslide<3->
  
  \ueb{Work in progess}
  \begin{itemize}
  \item Matthias will present recent work on the formalisation of \alert{randomised meldable heaps}
  \item we have been working on a new (amortised) complexity proof (that should be easier to automate)
    of \alert{zip trees} (an improvement on skip lists)
  \item<4-> ongoing research suggests that a fully automated analysis of \BLUE{Coupon Collector problem} or \BLUE{randomised quicksort} may belong to SF or even fantasy \dots
  \end{itemize}
\end{frame}

\section{Pilot Study: Splay Trees}
\begin{frame}[fragile]{Motivating Example: \only<1-4>{Splay Trees}\only<5->{Randomised Splay Trees}}
\vspace{-5mm}
\begin{minipage}[t][3cm][t]{\textwidth}
\begin{center}
\begin{onlyenv}<1>
  \begin{itemize}
  \item \yellowemph{splay trees} are self-adjusting data structures, no explicit balancing
    %clean and simple implementation
  \item basic operation is called \yellowemph{splaying}, called for insertion, look-up and removal
  \end{itemize}
\end{onlyenv}
\begin{onlyenv}<2-4>
\begin{lstlisting}[basicstyle=\color{basic}\ttfamily\small,lineskip=-1mm]
splay x t = match t with
 | node cl c cr -> match cl with
   | node bl b br -> match tick splay x bl with
     | node al a ar -> node al a (node ar b (node br c cr))
\end{lstlisting}
\end{onlyenv}
\begin{onlyenv}<5->
\begin{lstlisting}[escapeinside={!}{!},basicstyle=\color{basic}\ttfamily\small,lineskip=-1mm]
splay x t = match t with
 | node cl c cr -> match cl with
   | node bl b br -> match tick 1 2 splay x bl with
     | node al a ar -> if coin
       then tick 1 2 node al a (node ar b (node br c cr))
       else          node (node (node al a ar) b br) c cr
\end{lstlisting}
\end{onlyenv}
\end{center}
\end{minipage}
\vspace{-7mm}
\begin{center}
\begin{tikzpicture}[scale=1,
subtree/.style = {isosceles triangle, draw=black, dotted, shape border rotate=90, minimum width=7mm},
]
\begin{scope}
%\node[subtree] (8a) at (2,7.3) {} ;
\node (6a) at (2,6.5) {$c$} ;
\node (1a) at (1,6) {$b$} ;
\node (2a) at (0,5.5) {$a$} ;
\node[subtree] (3a) at (-1,4.4) {} ;
\node[subtree] (4a) at (1,4.4) {} ;
\node[subtree] (5a) at (2,5) {} ;
\node[subtree] (7a) at (3,5.5) {} ;

%\node at (8a.center) {$t$};
\node at (3a.center) {$al$};
\node at (4a.center) {$ar$};
\node at (5a.center) {$br$};
\node at (7a.center) {$cr$};

%\path[->] (8a.south) edge (6a);
\path[->] (6a) edge (1a);
\path[->] (6a) edge (7a.north);
\path[->] (1a) edge (2a);
\path[->] (1a) edge (5a.north);
\path[->] (2a) edge (3a.north);
\path[->] (2a) edge (4a.north);

%\node[subtree] (8b) at (7,7.3) {} ;
\node (2b) at (7,6.5) {$a$} ;
\node (1b) at (8,6) {$b$} ;
\node (6b) at (9,5.5) {$c$} ;
\node[subtree] (3b) at (6,5.5) {} ;
\node[subtree] (4b) at (7,5) {} ;
\node[subtree] (5b) at (8,4.5) {} ;
\node[subtree] (7b) at (10,4.5) {} ;

%\node at (8b.center) {$t$};
\node at (3b.center) {$al$};
\node at (4b.center) {$ar$};
\node at (5b.center) {$br$};
\node at (7b.center) {$cr$};

%\path[->] (8b.south) edge (2b);
\path[->] (2b) edge (3b.north);
\path[->] (2b) edge (1b);
\path[->] (1b) edge (4b.north);
\path[->] (1b) edge (6b);
\path[->] (6b) edge (7b.north);
\path[->] (6b) edge (5b.north);

\path[->] (4,5.5) edge node[above] {\small zig-zig} (5,5.5) ;
\only<5->{
\path[->] (4,5.5) edge node[below] {\yellowemph{$p=\sfrac{1}{2}$}} (5,5.5) ;
}
\end{scope}
\end{tikzpicture}
\end{center}
\vspace{-2mm}
%\begin{itemize}
%\item Introduced by Sleator and Tarjan
%\item Binary search tree
%\item Strictly increasing inorder traversal
%\item Self-adjusting (accessed items move to root).
%\item Basic operations (insert, lookup, remove; all based on \lstinline|splay|)
%\item original code imperative, shown code due to Okasaki, also develop splay heaps, as heap data structure
\begin{itemize}
\item<3-> \phantom{expected amortised} \alt<3>{\yellowemph{worst-case}}{worst-case} cost in $O(|t|)$
\item<4-> 
  \phantom{expected} \alt<4>{\yellowemph{amortised}}{amortised} worst-case cost in $O(\log(|t|))$, ie.\
  \yellowemph{$\sfrac{3}{2}$}\tikzanchor[xshift=-5mm, yshift=1mm]{a} ${} \cdot \log(|t|)$
  \callout<6>[callout=blue!5]{a}{210}{10mm}{\alert{\footnotesize due to Schoenmakers}}
\item<5->
  \yellowemph{expected amortised}  worst-case cost in $O(\log(|t|))$,
  ie.\ \yellowemph{$\sfrac{9}{8}$}\tikzanchor[xshift=-5mm, yshift=1mm]{b} ${} \cdot \log(|t|)$
  \callout<7>[callout=blue!5]{b}{210}{10mm}{\alert{\footnotesize due to \atlas}}
  % 25 % speed-up
\end{itemize}
  
% \begin{mybox}{Advantages}
%   \begin{itemize}
%   \item conceptionally simple operations, which are simple to implement
%   \item small memory foot print
%   \end{itemize}
% \end{mybox}
\end{frame}


\begin{frame}[fragile]{Automation}
\begin{enumerate}
\item set up type systems, such that \yellowemph{logarithmic} costs become
  representable
\item type inference induces constraints 
\item solve constraints
\end{enumerate}
\vspace{-.25cm}
\begin{center}
\begin{tikzpicture}[scale=0.25,stack/.style={rectangle split, rectangle split parts=#1,anchor=center}]

\begin{scope}[local bounding box=group]

%big wheels
\begin{scope}[scale=1.5, yshift=1.5cm, xshift=8cm]\lightwheel\end{scope}
\begin{scope}[scale=1.5, xshift=4cm]\darkwheel\end{scope}

% tools
\begin{scope}[xshift=4.6cm, yshift=3.8cm]\tool{}\end{scope}
\begin{scope}[xshift=8.2cm, yshift=2.3cm]\tool{}\end{scope}
\begin{scope}[xshift=8.5cm, yshift=-2.3cm]\tool{}\end{scope}
\begin{scope}[xshift=10.8cm, yshift=-1.2cm]\tool{}\end{scope}
\end{scope}

\node (input) [
  text width=2.5cm,
  left=1.5cm of group.west,
  align=center,
] {Source Code} ;

\draw[->] (input.east) -- (group.west);

\node[
  stack=3,
  text width=2.5cm,
  rectangle split,
  align=center,
  right=1.5cm of group.east,
] (output) {
\nodepart{one}  Amortised Complexity
\nodepart{two} \emph{or}
\nodepart{three} Unknown};
\draw[->] (group.east) --  (output.one west);
\draw[->] (group.east) -- (output.three west);
\end{tikzpicture}
\end{center}

\begin{center}
\url{github.com/lorenzleutgeb/atlas}\\[5mm]

\begin{tabular}{c@{\qquad}c}
  \includegraphics[height=1.5cm,width=1.5cm]{aec-badge_f_a_r.pdf}
  &
  \includegraphics[height=1.5cm,width=1.2cm]{1-available.pdf}
  \\
2021 &
2022
\end{tabular}
\end{center}
\end{frame}

\begin{frame}
\frametitle{Improvements over existing pen-and-paper proofs}
\vspace{-.7cm}  
\begin{center}  
{\small
  \begin{tabular}{|p{13ex}|l|l|l|}    
    \hline
    & \multicolumn{1}{|l}{%
        \atlas %\cite{LMZ:2021,LMZ:2022}
      }
    & \multicolumn{1}{|l}{%
        manual %\cite{AlbersK02,GambinM98,conf/cocoa/IaconoY16,Schoenmakers92}
      }
    & \multicolumn{1}{|l|}{%
        semi-automated %\cite{NipkowB19}
      }
    \\
    \hline\hline
    \multicolumn{4}{|c|}{Randomised Meldable Heaps}
    \\
    \hdashline
    \flst{insert} & $\log_2(|h|) + 1$ & $\bO(\log_2(|h|)$ & \_
    \\[1mm]
    \flst{delete\_min} & $2 \log_2(|h|)$ & $\bO(\log_2(|h|)$\tikzanchor[xshift=1mm,yshift=1mm]{a} 
    \\[1mm]
    \flst{meld} & $\log_2(|h_1|) + \log_2(|h_2|)$ & $\bO(\log_2(|h_1|)+ \log_2(|h_2|))$ & \_
    \\[1ex]
    \hline\hline
    \multicolumn{4}{|c|}{Paring Heaps}
    \\
    \hdashline
    \flst{merge\_pairs} & $\sfrac{3}{2} \log_2(|h|)$ & $\bO(\log_2(\size{h}))$ & $3\log_2(|h|)+4$
    \\[1mm]
    \flst{insert} & $\sfrac{1}{2}\log_2(\size{h})$\tikzanchor[xshift=5mm]{b}
    & $\bO(1)$ & $\log_2(\size{h}+1)+1$\tikzanchor[xshift=-5mm]{c} 
    \\[1mm]
    \flst{merge} & $\sfrac{1}{2}\log_2(\size{h_1} + \size{h_2})+1$ & $\bO(1)$ & $\log_2(\size{h_1} + \size{h_2}+1)+2$
    \\[1mm]
    \flst{del\_min} & $\log_2(\size{h})$ & $\bO(\log_2(\size{h}))$ &  $3 \log_2(\size{h}+1)+4$
    \\[1ex]
    \hline\hline
  \end{tabular}}
\end{center}
\vspace{-0.3cm}
\callout<2>[callout=blue!5]{a}{180}{2mm}{\footnotesize Gambin \& Malinowski}
\callout<3>[callout=blue!5]{b}{180}{2mm}{\footnotesize Iacono, very recently Sinnamon and Tarjan}
\callout<4>[callout=blue!5]{c}{180}{2mm}{\footnotesize Nipkow et al.}
% randomised meldable heaps, pen-and-paper proof due to Gambin and Malinowski
% pairing heaps, due to Iacono, very recent work by Sinnamon and Tarjan, sim
\onslide<5->
\end{frame}

\begin{frame}
\vspace{-.7cm}    
%\frametitle{Improvements (cont'd)}
\begin{center}  
{\small
  \begin{tabular}{|p{13ex}|l|l|l|}    
    \hline
    & \multicolumn{1}{|l}{%
        \atlas %\cite{LMZ:2021,LMZ:2022}
      }
    & \multicolumn{1}{|l}{%
        manual %\cite{AlbersK02,GambinM98,conf/cocoa/IaconoY16,Schoenmakers92}
      }
    & \multicolumn{1}{|l|}{%
        semi-automated %\cite{NipkowB19}
      }
    \\
    \hline\hline
    \multicolumn{4}{|c|}{Splay Trees}
    \\
    \hdashline
    \flst{insert} & $2 \log_2(|t|) + \sfrac{3}{2}$ & $2\log_2(\size{t}+1)+\bO(1)$ & $2 \log_2(|t|) + \sfrac{3}{2}$
    \\[1mm]
    \hfill {\footnotesize (randomised)} & $\sfrac{3}{4} \log_2(|t|) + \sfrac{3}{4} \log_2(|t|+1)$ & \_ & \_
    \\[1mm]
    \flst{delete} & $\sfrac{5}{2} \log_2(|t|) + 3$ & $3\log_2(\size{t}+1)+\bO(1)$ & $3 \log_2(|t|) + 2$
    \\[1mm]
    \hfill {\footnotesize (randomised)} & $\sfrac{9}{8} \log(|t|)$ & \_ & \_
    \\[1mm]
    \flst{splay} & $\sfrac{3}{2} \log_2(|t|)$ &  \alt<1-3>{$\sfrac{3}{2}\log_2(\size{t})+1$}{\yellowemph{$\sfrac{3}{2}\log_2(\size{t})+1$}} & $\sfrac{3}{2} \log_2(|t|) + 1$
    \\[1mm]
    %GM: Lemma 2: recursion costs 1/2, rotation 1/2, probability 1/2
    \hfill {\footnotesize (randomised)} & $\sfrac{9}{8} \log_2(|t|)$ & \alt<1-3>{$\sfrac{3}{4} + \sfrac{9}{4} \log_2(|t|)$}{\yellowemph{$\sfrac{3}{4} + \sfrac{9}{4} \log_2(|t|)$}}\tikzanchor[xshift=2mm,yshift=1mm]{a} & \_
    \\[1ex]
    \hline\hline
    \multicolumn{4}{|c|}{Splay Heap}
    \\
    \hdashline
    \flst{insert} & $\sfrac{1}{2} \log_2(|h|) + \log_2(|h|+1) + \sfrac{3}{2}$ & \_  &  $3\log_2(\size{h}+2) + 1$
    \\[1mm]
    \hfill {\footnotesize (randomised)} & $\sfrac{3}{4} \log_2(|h|) + \sfrac{3}{4} \log_2(|h|+1)$  & \_ & \_
    \\[1mm]
    \flst{delete\_min}  & $\log_2(|h|)$ & \_ & $2\log_2(\size{h}+1)+1$\tikzanchor[xshift=-13mm,yshift=-3mm]{c} 
    \\[1mm]
    \hfill {\footnotesize (randomised)} & $\sfrac{3}{4} \log_2(|h|)$ & \_ & \_
    \\
    \hline    
\end{tabular}}
\end{center}
\vspace{-0.3cm}
\callout<2>[callout=blue!5]{a}{180}{2mm}{\footnotesize Albers et al.}
\callout<3>[callout=blue!5]{c}{90}{2mm}{\footnotesize Nipkow et al.}
\onslide<4->
\end{frame}


\section{Methodology}

\begin{frame}{Amortised Analysis: Potential Method}
% \phi ... potential function
% a_f ... amortized cost
% c_f ... cost
% f ... operation

Introduced to analyse asymptotic behaviour over \alert{sequences} of operations.

The original formulation by Sleator and Tarjan%
\footnote{D.~Sleator and R.~Tarjan. Self-adjusting binary search trees.
  {\em JACM}, 32(3):652--686, 1985;
  R.~Tarjan. Amortized computational complexity. {\em SIAM J.~Alg.\ Disc.\ Meth}, 6(2):306--318, 1985.
}
%
\begin{align}
  \red{a_f(t)} &= \green{c_f(t)} + \blue{\phi(f(t))} - \red{\phi(t)}
  \\
%\red{\phi(t) + a_f(t)} &= \green{c_f(t)} + \blue{\phi(f(t))}
\intertext{is generalised to}
\red{\psi(t)} &\geq \green{c_f(t)} + \blue{\phi(f(t))} \\
  \red{\psi(t)} - \blue{\phi(f(t))} &\geq \green{c_f(t)} \label{eq:bound}
\end{align}

%\vspace{5mm}

%{\small{(choose $\red{\psi(t)} \coloneqq \red{a_f(t)} + \red{\phi(t)}$ to get back up)}}

\end{frame}

\begin{frame}{Logarithmic Amortised Analysis}
\begin{itemize}
\item \yellowemph{logarithmic} means that $\green{a_f(t)}$ is in $O(\log(\size{t}))$.
\item<2-> We expect the definition of $\phi$, $\psi$ to involve logarithmic terms
\begin{align}
\text{Size:}&&\size{\leaf} &\defsym 1&
\size{\tree{l}{\cdot}{r}} &\defsym \size{l} + \size{r}\\
\text{Rank:}&&\rk(\leaf) &\defsym 1&
\rk(\tree{l}{\cdot}{r}) &\defsym \rk(l) + \log(\size{l}) + \log(\size{r}) + \rk(r)
\end{align}

\item<3-> Potential functions we studied are adapted from Schoenmakers%
    \footnote{B.~Schoenmakers. A systematic analysis of splaying. {\em {IPL}}, 45(1):41--50, 1993.}
    
\begin{equation}
q_{*} \cdot \rk(t) + \sum_{a, b \in \N} q_{(a, b)} \cdot \alt<1-4>{\log(a \cdot \size{t} + b)}{\underbrace{\log(a \cdot \size{t} + b)}_{=: p(a,b)}}
\end{equation}

\item<4-> E.g.\ for \lstinline|splay| we get:
\vspace{-.5cm}
  \begin{alignat}{2}
\red{\psi(t)} &= \sfrac{3}{4}\rk(t) + \sfrac{3}{4} + \sfrac{9}{8} \log(\size{t}) \\
\blue{\phi(t)} &= \sfrac{3}{4}\rk(t) + \sfrac{3}{4}
\end{alignat}
\end{itemize}

\onslide<5->
\end{frame}

\begin{frame}
  \frametitle{Type-Based Expected Amortised Cost Analysis}
\begin{itemize}
\item \alert<3>{strict, first-order functional language.}
\item \alert<3>{necessarily simple, sufficiently powerful.}
\item{high-level architecture: generate constraints (via typing rules), solve.}
\end{itemize}

\onslide<2->
\begin{theorem}[Soundness]
\centering
\smallskip  
If
\uncover<3->{$\alt<3>{e\hspace{8ex}}{\eval{\uncover<4->{\sigma}}{\uncover<5->{c}}{e}{\uncover<4->{\mu}}}$}
and
$\onslide<6->{\Gamma \onslide<8->{{\mid} \red{Q}} \vdash} \onslide<3->{e} \onslide<6->{: \alpha} \onslide<8->{{\mid} \blue{Q'}}$
then
$\onslide<7->{\underbrace{\Phi(\Gamma\sigma\onslide<8->{{\mid} \red{Q}})}_{\red{\psi(t)}} \geqslant} \onslide<5->{c}
  \onslide<7->{+ \Expect{\mu}{\lambda v. \underbrace{\Phi(v \onslide<8->{{\mid} \blue{Q'}})}_{\blue{\phi(f(t))}}}}$.
\end{theorem}
 
Ingredients:
\begin{itemize}
\item<3-> \alert<3>{syntax}
\item<4-> operational big-step semantics \alert<4>{$\eval{\sigma}{\onslide<5->{c}}{e}{\mu}$, $\mu$ distribution of final states} \onslide<5->{with \alert<5>{cost} measure}
\item<6-> type system $\vdash$ \onslide<8->{with \alert<8>{annotated} types}
\item<9-> \emph{read:} given a typing ${\Gamma {{\mid} \red{Q}} \vdash} {e: \alpha}{{\mid} \blue{Q'}}$ the theorem holds for
  \emph{all substitutions} $\sigma$    
\end{itemize}
\end{frame}

\begin{frame}
  \frametitle{Operational Semantics%
    \footnote{%
    Taking (a lot) inspiration from Borgström et al. A lambda-calculus foundation for universal probabilistic programming, ICFP'16 and Wang et al. Raising expectations: automating expected cost analysis with types, ICFP'20.}}
  
\onslide<1->{What is ``cost''?}
\onslide<2->{\begin{mathpar}
\infer
[(tick)]
  {\evaln{\sigma}{n+1}{c+\prob{\mu} \cdot \sfrac{a}{b}}{\tick[$a$][$b$]{e}}{\mu}}
  {\evaln{\sigma}{n}{c}{e}{\mu}}
\end{mathpar}}

\onslide<1->{How does ``cost'' compose?}
\onslide<3->{\begin{mathpar}
\infer
[(let)]
  {\evaln{\sigma}{n+1}{c_1 + \sum_{w \in \supp{\nu}} \nu(w) \cdot c_w}{\flstk{let }x\flst{ = }e_1\flstk{ in }e_2}{\sum_{w \in \supp{\nu}} \nu(w) \cdot \mu_w}}%
  {%
  \evaln{\sigma}{n}{c_1}{e_1}{\nu}
  &
  \text{for all $w \in \supp{\nu}$:
  $\evaln{\sigma[x \mapsto w]}{n}{c_w}{e_2}{\mu_w}$
  }
  }
\end{mathpar}}
\end{frame}

\begin{frame}[fragile]{Example: Descending a Binary Tree}
\begin{center}
\begin{lstlisting}[escapeinside={!}{!},basicstyle=\color{basic}\ttfamily]
descend t = match t with
  | leaf       -> leaf
  | node l a r -> if coin
    then let xl = tick descend l in node xl a  r     (* !\boxemph{$\swarrow$}! *)
    else let xr = tick descend r in node  l a xr     (* !\boxemph{$\searrow$}! *)
\end{lstlisting}
\end{center}

\end{frame}
\begin{frame}[fragile]{Example: Descending a Binary Tree}
\vspace{-1cm}
\begin{equation*}
\infer
  [\onslide<3->{\rulematch}]%
  {\tjudge{\typed{t}{\TreeShort}}{Q}{\textup{\lstinline[keepspaces=true]!match t with | leaf -> leaf | node l a r -> ! \boxemph{$\swarrow{}\!\!\!\!\searrow$}}}{\TreeShort}{Q'}}%
  {\onslide<3->{%
    \infer%
    [\onslide<4->{\rulew}]%
    {\tjudge{\typed{l}{\TreeShort}, \typed{r}{\TreeShort}}{Q_1}{\textup{\lstinline!if coin then! \boxemph{$\swarrow$} \lstinline!else! \boxemph{$\searrow$}}}{\TreeShort}{Q'}}%
    {\onslide<4->{%
		  \infer%
        [\onslide<6->{\rulecoin}]%
        {\tjudge{\typed{l}{\TreeShort}, \typed{r}{\TreeShort}}{Q_2}{\textup{\lstinline!if coin then! \boxemph{$\swarrow$} \lstinline!else! \boxemph{$\searrow$}}}{\TreeShort}{Q'}}%
        {\onslide<6->{%
          \infer%
            [\onslide<7->{\rulelet}]%
            {\tjudge{\typed{l}{\TreeShort}, \typed{r}{\TreeShort}}{Q_3}{\textup{\lstinline|let xl = tick descend l in node xl a r|}}{\TreeShort}{Q'}}
            {\onslide<7->{%
              \infer%
                [\onslide<8->{\ruletick}]%
                {\tjudge{\typed{l}{\TreeShort}}{Q_4}{\textup{\lstinline|tick descend l|}}{\TreeShort}{Q_6}}
                {\onslide<8->{%
                  \infer%
                    [\onslide<9->{\ruleapp}]%
                    {\tjudge{\typed{l}{\TreeShort}}{Q_5}{\textup{\lstinline|descend l|}}{\TreeShort}{Q_6}}%
                    {\onslide<9->{\typed{\textup{\lstinline{descend}}}{\atypdcl{\TreeShort}{Q}{\TreeShort}{Q'}}}}
                }}
                &
                \tjudge{\typed{x_l}{\TreeShort},\typed{r}{\TreeShort}}{Q_7}{\textup{\lstinline|node xl a r|}}{\TreeShort}{Q'}
            }}
        }}
    }}
  }}
\end{equation*}
\vspace*{-5mm}
\begin{minipage}[t][2cm][t]{\textwidth}  
\alt<1-2>{%
Heuristically choose shape of potential functions, then solve:\tikzanchor[xshift=0mm, yshift=1mm]{c}
\begin{align*}  
  \red{\potential{\typed{t}{\Tree}}{Q}} & \defsym \GRAY{q_\ast} \cdot \rk(t) + \GRAY{q_{(1,0)}}
  \cdot p_{(1,0)}(t) +
                                        \GRAY{q_{(0,2)}}\cdot p_{(0,2)}(t)\\
\blue{\potential{\typed{t}{\Tree}}{Q'}} & \defsym \GRAY{q'_\ast} \cdot \rk(t)
      + \GRAY{q'_{(1,0)}} \cdot p_{(1,0)}(t) + \GRAY{q'_{(0,2)}} \cdot p_{(0,2)}(t)
\end{align*}
}%
{\alt<3>{Distribute potential of \lstinline|t| to \lstinline|l| and \lstinline|r|:%
\begin{align*}
q^{(1)}_1  = q^{(1)}_2 = q_\ast && q^{(1)}_{(1,1,0)}  = q_{(1,0)} && q^{(1)}_{(1,0,0)} = q^{(1)}_{(0,1,0)} = q_\ast && q^{(1)}_{(0,0,2)} = q_{(0,2)}
\end{align*}
}%
{\alt<4-5>{We can weaken/adapt/rearrange coefficients, given
\begin{align}
\Phi(\sigma; \typed{l}{\TreeShort}, \typed{r}{\TreeShort}; Q_1) \geqslant \Phi(\sigma; \typed{l}{\TreeShort}, \typed{r}{\TreeShort}; Q_2) \qquad \text{for all valuations $\sigma$}
\end{align}
which we prove by using facts about $\log$\tikzanchor[xshift=-5mm, yshift=2mm]{c}
and Farkas' lemma.\tikzanchor[xshift=0mm, yshift=1mm]{a}
}%
{\alt<6>{Formalisation of a coin toss with $p=\sfrac{1}{2}$. Distribution of potential $Q_2 = p \cdot Q_3 + (1-p) \cdot Q_4$:
%
\begin{align*}
  q^{(2)}_{(0,0,2)} & = \sfrac{1}{2} \cdot q^{(3)}_{(0,0,2)} + \sfrac{1}{2} \cdot q^{(4)}_{(0,0,2)} &
  q^{(2)}_{(1,0,0)} & = \sfrac{1}{2} \cdot q^{(3)}_{(1,0,0)} + \sfrac{1}{2} \cdot q^{(4)}_{(1,0,0)} \\
  q^{(2)}_{(0,1,0)} & = \sfrac{1}{2} \cdot q^{(3)}_{(0,1,0)} + \sfrac{1}{2} \cdot q^{(4)}_{(0,1,0)}
\end{align*}
}%
{\alt<7>{Most involved rule of the type system. Roughly:
\begin{alignat*}{7}
Q_3 & \ & = & \ & Q_4    & \ + \ && \Delta & \ &   & \ &     & \quad & \text{Input potential ($Q_3$) covers recursive call ($Q_4$) and remainder ($\Delta$).}\\
    & \ &   & \ & \Delta & \ + \ && Q_6    & \ & = & \ & Q_7 & \quad & \text{Remainder ($\Delta$) and leftover ($Q_6$) cover result ($Q_7$).}
\end{alignat*}
}%
{\alt<8>{Ensure that costs are accounted for:

\begin{align*}
Q_4 = Q_5 + 1 
\end{align*}
}%
{\alt<9>{Verify that recursive call is well-typed:
\begin{align*}
Q_5 &= Q \\
Q_6 &= Q'
\end{align*}
}{Other}}}}}}}
\end{minipage}
\vspace*{-5mm}
\callout<2>[callout=blue!5]{c}{180}{5mm}{\alert{\footnotesize and set $q_{(1,0)}=1$, only}}
\callout<5>[callout=blue!5]{a}{180}{10mm}{\alert{\footnotesize essential for Z3}}
\callout<5>[callout=blue!5]{c}{270}{5mm}{\alert{\footnotesize eg. $2 \cdot \log(x+y) \geqslant \log(x)+\log(y)+2$}}

\end{frame}

\begin{frame}{Soundness: Small-Step}
\begin{equation*}
\infer[\alert<1,5>{\ruletick}]{%
  \tjudge{\Gamma}{Q+\sfrac{a}{b}}{\textup{\lstinline!tick a b!}\ e}{\alpha}{Q'}
}{%
  \tjudge{\Gamma}{Q}{e}{\alpha}{Q'}
}
\end{equation*}

\begin{theorem}[Soundness for \alert<1>{$\ruletick$}] \label{t:1}
Let $\Program$ be well-typed. Suppose $\tjudge{\Gamma}{Q}{e}{\alpha}{Q'}$ and $e\sigma \tomulti{c}_\infty\tikzanchor[xshift=-5mm,yshift=-2mm]{a} \mu$.
Then $\spotential{\Gamma}{Q} \geqslant c + \Expect{\mu}{\lambda v. \potential{v}{Q'}}$.
%Further, if $\tjudgecf{\Gamma}{Q}{e}{\alpha}{Q'}$, then $\spotential{\Gamma}{Q} \geqslant \Expect{\mu}{\lambda v. \potential{v}{Q'}}$.
\end{theorem}

\begin{itemize}
  \item<3-> semantics considers cost on \alert<3->{all paths}.
  \item<4-> typing \alert{does imply} positive almost-sure termination.
  \item<5-> generated constraints are inflexible
  \item<5-> randomised splay trees cannot be handled
\end{itemize}

\callout<2>[callout=blue!5]{a}{90}{2mm}{\footnotesize based on probabilistic ARSs.}
\end{frame}

\begin{frame}{Soundness: Big-Step}
\begin{equation*}
\infer[\alert<1>{\ruletickast}]{%
  \tjudge{\Gamma}{Q}{\textup{\lstinline!tick a b!}\ e}{\alpha}{Q'-\sfrac{a}{b}}
}{%
  \tjudge{\Gamma}{Q}{e}{\alpha}{Q'}
}
\end{equation*}

\begin{theorem}[Soundness for \alert<1>{$\ruletickast$}] \label{t:2}
Let $\Program$ be well-typed.
Suppose $\tjudge{\Gamma}{Q}{e}{\alpha}{Q'}$ and $\eval{\sigma}{c}{e}{\mu}$.
Then, we have $\spotential{\Gamma}{Q} \geqslant c + \Expect{\mu}{\lambda v. \potential{v}{Q'}}$.
%Further, if $\tjudgecf{\Gamma}{Q}{e}{\alpha}{Q'}$, then $\spotential{\Gamma}{Q} \geqslant \Expect{\mu}{\lambda v. \potential{v}{Q'}}$.
\end{theorem}

\begin{itemize}
  \item<2-> big-step semantics considers cost on \alert<1>{all terminating paths}.
  \item<3-> typing \alert{does not imply} termination.
  \item<4-> generated constraints are more flexible, i.e.\ more programs can be typed.
  \item<5-> can be combined with separate termination analysis.
\end{itemize}

\end{frame}

\begin{frame}
  \frametitle{Conclusions}
  
  \begin{itemize}
  \item fully automated analysis of (expected) cost analysis of data structures, like
    (randomised) splay trees, splay heaps, and pairing heaps and
    randomised melding heaps.
  \item bounds match or improve literature (sophisticated pen-and-paper proofs by Schoenmakers and \textsf{Isabelle/HOL} proofs by Nipkow and Brinkop).
  \item<2-> pen-and-paper proofs may incorporate functional correctness arguments, we could dispense with.
  \item<2-> pen-and-paper proofs may run out of steam, due to the tedious calculations necessary; \atlas\ pervails
  \end{itemize}
  
\end{frame}

\begin{frame}
  \frametitle{Summary%
    \footnote{This work is dedicated to Martin Hofmann, whose tragic accident happened before we could finalise the results.}
  }
  \vspace{-1ex}
  Automated amortised analysis achieved by:
  \begin{itemize}
  \item user fixes template potential functions.
  \item constraints over template function coefficients are generated.
  \item optimising constraint solver searches for 'small' coefficients.
  \end{itemize}

  \medskip
  
  \begin{mybox}{Vision}
    New method for data structure analysis. Extend to further types of data structures (for which different potential functions are needed).
  \end{mybox}

  \medskip
  \onslide<2->

  \ueb{Remark}
  \begin{itemize}
  \item this vision seems to have gained traction (see Lorenz's talk)
  \end{itemize}
\end{frame}  

\begin{frame}
\frametitle{Observations}
\vspace{-1mm}
\begin{mybox}{A}
  data structures are typically formulated based on \textbf{invariants}
  \begin{itemize}
  \item binomial heaps
  \item Fibonacci heaps
  \item (weak) AVL trees
  \end{itemize}
  %
  how to capture these invariants and exploit them in type inference?
\end{mybox}
\vspace{-1mm}
\onslide<2->
\begin{mybox}{B}
    \begin{itemize}
    \item extend probabilistic language feature, eg. dynamic sampling instructions
    \item \textbf{zip trees} and friends are of interest
    \end{itemize}
\end{mybox}
\vspace{-1mm}
\onslide<3->
\begin{mybox}{C}
  capture lazy evaluation, and thus eg.\ \textbf{lazy skew heaps}.
\end{mybox}

\end{frame}  

\section{Thank You for Your Attention!}

% \bibliographystyle{plain}
% \bibliography{slides}
\end{document}



